%-------------------------------------------------------------------------------
%	SECTION TITLE
%-------------------------------------------------------------------------------
\cvsection{Expérience}


%-------------------------------------------------------------------------------
%	CONTENT
%-------------------------------------------------------------------------------
\begin{cventries}

%---------------------------------------------------------
  \cventry
    {Assistant développeur Web \& Intégration de données (Stage)} % Job title
    {Click \& Mortar (C\&M)} % Organization
    {Montréal, Québec} % Location
    {Hiver 2024} % Date(s)
    {
      \begin{cvitems} % Description(s) of tasks/responsibilities
      	\item {Intégration hebdomadaire des recettes de Tada! dans Sitecore pour Sobeys, incluant sélection, saisie et contrôle qualité des données.}
      	\item {Résolution d’un problème de suivi de clics sur le site du Théâtre Duceppe sans accès au code source, en utilisant Google Tag Manager et des sélecteurs CSS.}
      	\item {Conception et mise en place d’un tableau de bord dynamique pour l’agence, combinant Google Analytics, Google Tag Manager et Looker Studio pour suivre la santé des sites et la performance des campagnes.}
      	\item {Analyse des performances web pour plusieurs clients (Théâtre Duceppe, BMR, Skyspa Forena) et proposition de recommandations pour optimiser les campagnes publicitaires et le SEO.}
      	\item {Collaboration efficace avec un coéquipier stagiaire : répartition des tâches, travail autonome et revue mutuelle pour garantir la qualité et l’efficacité des livrables.}
      \end{cvitems}
    }

%---------------------------------------------------------
  \cventry
    {Développeur d’applications web (Stage)} % Job title
    {Progression} % Organization
    {Montréal, Québec} % Location
    {Hiver 2023} % Date(s)
    {
      \begin{cvitems}
  		\item {Implémentation d’un système complet d’authentification par courriel.}
  		\item {Utilisation de GitLab pour la gestion de version, les revues de code et le suivi des tickets.}
  		\item {Participation aux mêlées quotidiennes et au suivi Kanban dans un environnement Agile.}
  		\item {Développement d’interfaces dynamiques avec Vue.js et Vuex; intégration au backend Laravel (PHP).}
   	  \end{cvitems}
    }
    
%---------------------------------------------------------

  \cventry
    {Étudiant en informatique} % Titre
    {Ministère des Transports et de la Mobilité durable(MTMD)} % Organisation
    {Montréal, Québec} % Lieu
    {Juin 2019 -- Octobre 2021} % Dates
    {
	  \begin{cvitems}
		\item {Suivi et gestion des produits SIMDUT et inventaires des bâtiments ministériels dans l'ensemble des régions administrative du Québec, à l’aide d’outils Excel, avec un système d’alerte interne pour les mises à jour des produits ayant dépassé leurs dates de péremption.}
		\item {Développement d’un prototype sous Microsoft Access pour la gestion de la flotte de véhicules du Québec.}
	  \end{cvitems}
    }
    
%---------------------------------------------------------
\end{cventries}